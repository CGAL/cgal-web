\documentclass[12pt,a4paper]{article}

\pagestyle{empty} % no page number
\parskip 16pt     % space between paragraphs
\parindent 0pt    % indent for new paragraph

\usepackage{geometry}
\geometry{left=0.5in,top=0.5in,right=0.5in,bottom=0.5in} %margins
\usepackage{graphicx}
\usepackage{color,framed}
\usepackage{calc}
\usepackage{alltt,url}
\usepackage{bold-extra}

\DeclareGraphicsExtensions{.png}
\DeclareGraphicsExtensions{.jpg}
\DeclareGraphicsRule{.png}{eps}{.bb}{`convert -compress JPEG #1 eps2:-}
\DeclareGraphicsRule{.jpg}{eps}{.bb}{`convert #1 eps2:-}

\newcommand{\cgal}{\textsc{Cgal}}
\linespread{1.2}

\begin{document}
\begin{framed}
  \newlength{\cgalwidth}\setlength{\cgalwidth}{\textwidth}
  \begin{center}
    \includegraphics[width=\cgalwidth]{GSoC2011.pdf}

    \textbf{\Huge Looking for a Cool Summer Job?}
  \end{center}

  \textbf{\large The Computational Geometry Algorithm Library (\cgal{})
    open-source project, has been accepted to participate in the 2011 Google
    Summer of Code (GSoC) program; see \url{http://www.google-melange.com/}.
    This means that Google is supporting \cgal{} by funding some internships
    potentially from all over the world, for writing code under the guidance
    of experienced developers, so called mentors.}

  \textbf{\large If you are a smart user, have the time, and would like to
    contribute and to get paid for that, you should apply! If you have
    some master or PhD students who can spend their summer (full time)
    on a nice \cgal{} project, please advise them to apply.}

  \textbf{\large We have already listed several ideas for projects and
    instructions to applicants at
    \url{http://www.cgal.org/project_ideas.html}. If you find a project
    that suits your expertise and interest, contact the mentor of the
    project to get acquainted with her or him and to obtain more details.}

  %\centerline{The \cgal{} team}
  \includegraphics[width=\cgalwidth]{../../images/cgal-full.pdf}
  
\end{framed}

\end{document}
